% commands
%%%%%%%%%%%%%%%%%%%%%%%%%%%%%%%%%%%%%%%%%%%%%%%%%%%%%%%%%%%%%%%%%%%%%%%%%%%%%%%%%%%%
% Einrückung global entfernen
\setlength{\parindent}{0pt}

%%%%%%%%%%%%%%%%%%%%%%%%%%%%%%%%%%%%%%%%%%%%%%%%%%%%%%%%%%%%%%%%%%%%%%%%%%%%%%%%%%%%
% maximal zulässigen Wortabstand einstellen
\setlength\emergencystretch{1em} 

%%%%%%%%%%%%%%%%%%%%%%%%%%%%%%%%%%%%%%%%%%%%%%%%%%%%%%%%%%%%%%%%%%%%%%%%%%%%%%%%%%%%
% Farben fuer Verlinkungen definieren/aendern
 \usepackage{color}
   %Linkfarbe setzen
   \definecolor{LinkColor}{gray}{0}
   %PDF Farben einstellen
   \hypersetup{colorlinks=true,%
   linkcolor=LinkColor,%
   citecolor=LinkColor,%
   filecolor=LinkColor,%
   menucolor=LinkColor,%
   urlcolor=LinkColor} 
   
%%%%%%%%%%%%%%%%%%%%%%%%%%%%%%%%%%%%%%%%%%%%%%%%%%%%%%%%%%%%%%%%%%%%%%%%%%%%%%%%%%%%
% Aussehen der Ueberschriften

	\setkomafont{chapter}{\huge\sffamily}    % Chapter
	\setkomafont{sectioning}{\sffamily} %  % Titelzeilen % \bfseries
	\setkomafont{pagenumber}{\sffamily}             % Seitenzahl
	\setkomafont{descriptionlabel}{\itshape}        % Kopfzeile
	%
	\addtokomafont{sectioning}{\color{black}} % Farbe der Ueberschriften
	\addtokomafont{chapter}{\color{black}} % Farbe der Ueberschriften
	\renewcommand*{\raggedsection}{\raggedright} % Titelzeile linksbuendig, haengend

%%%%%%%%%%%%%%%%%%%%%%%%%%%%%%%%%%%%%%%%%%%%%%%%%%%%%%%%%%%%%%%%%%%%%%%%%%%%%%%%%%%%
% Silbentrennung
	\hyphenation{}
	
%%%%%%%%%%%%%%%%%%%%%%%%%%%%%%%%%%%%%%%%%%%%%%%%%%%%%%%%%%%%%%%%%%%%%%%%%%%%%%%%%%%%
% Feste Spaltenbreiten zentriert, linksbuendig oder rechtsbuendig	
\newcolumntype{L}[1]{>{\raggedright\arraybackslash}p{#1}} % linksbündig mit Breitenangabe
\newcolumntype{C}[1]{>{\centering\arraybackslash}p{#1}} % zentriert mit Breitenangabe
\newcolumntype{R}[1]{>{\raggedleft\arraybackslash}p{#1}} % rechtsbündig mit Breitenangabe	


% theorem environment
\theoremstyle{plain} % typical theorem-style (bold header, italic font)
\newtheorem{theorem}{Theorem}[section]
\newtheorem{lemma}[theorem]{Lemma}
\newtheorem{corollary}[theorem]{Korollar}
\newtheorem{observation}[theorem]{Beobachtung}
\newtheorem{proposition}[theorem]{Proposition}
\newtheorem{rrule}{Reduktionsregel}[section]

\crefname{rrule}{Reduktionsregel}{Reduktionsregeln} % how cleverref (\cref) has to handle these environments
\Crefname{rrule}{RR}{RRs} % with \Cref us can use what you defined here---somtimes useful if you want to abbreviate environment when citing (e.g. in tables or figures)

\theoremstyle{definition} % typical definition-style (bold header, normal font)
\newtheorem{definition}[theorem]{Definition}

\theoremstyle{remark} % typical theorem-style (italic header, normal font)
\newtheorem{example}{Example}
\newtheorem*{remark}{Remark}
\newtheorem{reduction}{Reduction}
\theoremstyle{plain}

% problem definition evironment: gets three arguments (1) problem name #1 (2) input specification #2 (3) question specification #3
\newcommand{\problemdef}[3]{
  \begin{center}
    \begin{minipage}{0.95\textwidth}
      \noindent
      \textsc{#1}
      
      \vspace{2pt}
      \setlength{\tabcolsep}{3pt}
      \begin{tabularx}{\textwidth}{@{}lX@{}}
        \textbf{Eingabe:} 		& #2 \\
        \textbf{Frage:} 	& #3
      \end{tabularx}
    \end{minipage}
  \end{center}
}



	
%%%%%%%%%%%%%%%%%%%%%%%%%%%%%%%%%%%%%%%%%%%%%%%%%%%%%%%%%%%%%%%%%%%%%%%%%%%%%%%%%%%%
% Macros 

% \NewDocumentCommand{\foocmd}{ O{default1} O{default2} m }{#1 #2 #3}
%                         %     ⤷ #1        ⤷ #2    ⤷ #3

%   \foocmd{foo} \par                           default1 default2 foo
%   \foocmd[nondefault1]{foo} \par              nondefault1 default2 foo
%   \foocmd[nondefault2][notfoo2]{foo} \par     nondefault12 notfoo2 foo


\newboolean{boolvar} %Deklaration
\setboolean{boolvar}{true} %Zuweisung

 \NewDocumentCommand{\todo}{  O{ueberarbeiten!} O{red}  }{
 \ifthenelse{\boolean{boolvar}}
    { [\textcolor{#2}{TODO: #1 }]  }
    { %[\textcolor{$2}{TODO: $1 }]  }

    }
} 

% \newcommand{\delaunayRefinement}{Delaunay-Refinement }
% \newcommand{\delaunayRefinements}{Delaunay-Refinements }
% \newcommand{\delaunayTriagnulierung}{Delaunay-Triagnulierung}

 
 \NewDocumentCommand{\innenwinkel}{ s O{i} O{j} O{k}  }{
 \IfBooleanTF #1 
 {\theta_{#2}^{#3,#4}}
 {$\theta_{#2}^{#3,#4}$} 
 }
 \NewDocumentCommand{\nominnenwinkel}{ s O{i} O{j} O{k}  }{
 \IfBooleanTF #1 
 {\tilde{\theta}_{#2}^{#3,#4}} 
 {$\tilde{\theta} _{#2}^{#3,#4}$}
 }
 
 \NewDocumentCommand{\gesamtwinkel}{s O{i}}{
   \IfBooleanTF #1 
 {\Theta_{#2}}
 {$\Theta_{#2}$} 
 }
 
 \NewDocumentCommand{\bezugsrichtung}{s O{i}}{
   \IfBooleanTF #1 
 {e_{#2}}
 {$e_{#2}$}
 }
 
  \NewDocumentCommand{\richtung}{s O{i}}{
   \IfBooleanTF #1 
 {\phi_{#2}}
 {$\phi_{#2}$}
 }
 
 
 
 \NewDocumentCommand{\algorithmusname}{s O{i}}{
   \IfBooleanTF #1 
 {\text{intrinsisches Delaunay-Refinement}}
 {intrinsisches Delaunay-Refinement}
 }
 
 
 %umkreisradius-kürzeste-kante-verhältnis 
 \NewDocumentCommand{\kuezesteKante}{s}{
    \IfBooleanTF #1 
        {d}
        {$d$}
 }
 \NewDocumentCommand{\umkreisradius}{s}{
     \IfBooleanTF #1 
        {r}
        {$r$}
 }
  \NewDocumentCommand{\kleinsterWinkel}{s}{
     \IfBooleanTF #1 
        {\Theta_{\text{min}}}
        {$\Theta_{\text{min}}$}
 }
 
   \NewDocumentCommand{\Dichte}{s O{v}}{
     \IfBooleanTF #1 
        {D_{#2}}
        {$D_{#2}$}
 }
 
 
    \NewDocumentCommand{\irregulaereKante}{s O{(v,v)}}{
     \IfBooleanTF #1 
        {e_{#2}}
        {$e_{#2}$}
 }
 
 
    \NewDocumentCommand{\regulaereKante}{s O{(v,s)}}{
     \IfBooleanTF #1 
        {e_{#2}}
        {$e_{#2}$}
 }
 
 \NewDocumentCommand{\UrzkK}{s O{(v,s)}}{
          \IfBooleanTF #1 
        {\kappa}
        {$\kappa$}
 }

 
 
 
 
 
 
 
 