\begin{titlepage}
	\sffamily
    \raggedleft
	\small
	\begin{center}		
		\color{gray}
		\begin{tabularx}{\textwidth}{R{0.5\textwidth}L{0.5\textwidth}}
			\includegraphics[width=0.25\textwidth]{images/Logo_TU.pdf}&\\
			&\textbf{Fakultät IV}\\
			&\textbf{Elektrotechnik und Informatik}\\
			&\textbf{Institut für Technische Informatik und Mikroelektronik}\\
			&\textbf{Fachgebiet Computer Graphics}\\
			&\textbf{Prof. Dr. Marc Alexa}\\
		\end{tabularx}	
	\end{center}
	\mbox{}\vspace{2\baselineskip}\\
	\sffamily\huge
	\centering   
	%TODO Bachlorarbeit größer
	Bachelorarbeit\vspace{0.25\baselineskip}\\
	%\vspace{2\baselineskip}\\
	\sffamily\Huge
Winkeldefektbedingungen für intrinsisches Delaunay-Refinement  \vspace{1.5\baselineskip}\\
	\sffamily\normalsize
	Vorgelegt von:\vspace{0.5\baselineskip}\\	
	Tarik Abdel-Moati Moussa Salama\vspace{0.25\baselineskip}\\
	Matrikelnummer: 380833\vspace{0.25\baselineskip}\\
	
	\sffamily\large
	Berlin, den \today
	\vspace{3\baselineskip}\\

   \vfill
   \raggedright
   \small 
   \centering
	\vfill	
	\uline{Erstgutachter:}\vspace{0.25\baselineskip}\\
	Prof. Dr. Marc Alexa (TU Berlin)\vspace{0.5\baselineskip}\\
	\uline{Zweitgutachter:}\vspace{0.25\baselineskip}\\
	Prof. Dr. Alexander I. Bobenko (TU Berlin)
	\vspace{2\baselineskip}\\
	\uline{Betreuer:}\vspace{0.25\baselineskip}\\
	Ugo Paavo Finnendahl (TU Berlin)\vspace{0.25\baselineskip}\\
	%Carl Lutz (TU Berlin)\\

\end{titlepage}
